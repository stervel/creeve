\section{phonology and orthography}
The following table shows the 19 consonant phonemes plus three allophones. Fortis consonants are always voiceless, and sometimes also glottalized in some back consonant, while lenis consonants are always unaspirated and un-glottalized, and generally fully voiced. The alveolars are usually apical, but can be laminal in some circumstances.

\begin{table}[]
\centering
\caption{Consonants}
\label{1a}
\begin{tabular}{llc|cccccc}
           &                                            &                   & \multicolumn{2}{c}{labial} & coronal       & palatal & velar & glottal \\ \hline
nasal      &                                            &                   & m            &             & n             &         &       &         \\
plosive    &                                            & {\scriptsize $+$} & (p)          &             & t             &         &       &         \\
           &                                            & {\scriptsize $-$} & b            &             & d             &         &       &         \\
affricates &                                            &                   &              &             & \SPL{\t{ts}}  &         &       &         \\
continuant & \multirow{2}{*}{\footnotesize sibilant}    & {\scriptsize $+$} &              &             & s             &         &       &         \\
           &                                            & {\scriptsize $-$} &              &             & z             &         &       &         \\
           & \multirow{2}{*}{\footnotesize non-sibilant}& {\scriptsize $+$} & \SPL{B}      & (f)           &               & \SPL{J} & x     &         \\
           &                                            & {\scriptsize $-$} & w            & v         &               & j       & \SPL{G}& h       \\
rhotic     &                                            & {\scriptsize $+$} &              &            & r              &         &       &         \\
           &                                            & {\scriptsize $-$} &              &     &  (\SPL{R})             &         &       &         \\
lateral    &                                            &                   &              &           & l              &         &       &        
\end{tabular}
\end{table}

\section*{allophony}
\begin{itemize}
\item Apart from alveolars and palatals, all fortis articulated before glides in a syllable. Fortition may occur on alveolars when articulated before glides in a syllable.
\item All coronals often be laminal alveolar [\SPL{\textsubsquare{n} \textsubsquare{t} \textsubsquare{d} \textsubsquare{s} \textsubsquare{z} \textsubsquare{\t{ts}}}] or laminal denti-alveolar [\SPL{\|[n \|[t \|[d \|[s \|[z]}
\item The stops /\SPL{p t}/ are slightly lenited and always occur before glides.
\item /\SPL{J x G}/ are usually labialized [\SPL{J\super w x\super w G\super w}]. Velar consonants are labialized weakly compared to palatal.
\item /p/ is always slightly voiced [\r p] and only occurs in loanwords or, occassionally, realized before glides.
\item All phonemes excepts glides itself always realized as fortis before glides.
\item /\SPL{R}/ always realized as alveolar apical.
\item /h/ often realized as voiced /\SPL{H}/, especially when intervocal.
\item Nasals are free form and dependent to the precedent place of articulation (but never go beyond coronal).
\end{itemize}

\section*{sound changes}
Intervocalic non-coronal and trill may be lenited before glides.

\begin{quote}
    \begin{sample}
        \smp{immivì}{Im:@wa}{of flowers}{\{\SPL{B r J x}\}$>$\{\SPL{w R j G}\}$/$V\_V}
\end{sample}
\end{quote}

\noindent Coronal flap always occur as coda of the syllable, and can be assimilated if the next phoneme is also coronal.

\begin{quote}
\begin{sample}
        \smp{larnè}{lAn:a}{change}{\SPL{R}$>$[coronal]\SPL{:}$/$\_\{\SPL{m n t d s z}\}}
\end{sample}
\end{quote}
